\documentclass[11pt,a4paper]{article}

% Packages - compatible with all online compilers
\usepackage[utf8]{inputenc}
\usepackage[T1]{fontenc}
\usepackage[margin=0.8in]{geometry}
\usepackage{listings}
\usepackage{xcolor}
\usepackage{hyperref}
\usepackage{fancyhdr}
\usepackage{titlesec}
\usepackage{enumitem}
\usepackage{float}
\usepackage{booktabs}
\usepackage{array}
\usepackage{longtable}
\usepackage{multicol}

% Colors
\definecolor{commentgreen}{RGB}{34,139,34}
\definecolor{stringcolor}{RGB}{208,76,239}
\definecolor{keywordcolor}{RGB}{0,0,255}
\definecolor{backgroundcolor}{RGB}{248,248,248}
\definecolor{numbercolor}{RGB}{128,128,128}
\definecolor{exercisecolor}{RGB}{255,250,205}
\definecolor{solutioncolor}{RGB}{230,255,230}
\definecolor{warningcolor}{RGB}{255,230,230}
\definecolor{tipcolor}{RGB}{230,240,255}
\definecolor{notecolor}{RGB}{255,245,220}
\definecolor{cheatcolor}{RGB}{240,248,255}

% SystemVerilog language definition
\lstdefinelanguage{SystemVerilog}{
  alsoletter={@,\$},
  morekeywords={
    module, endmodule, input, output, inout, wire, reg, logic, bit,
    always, always_ff, always_comb, always_latch, initial, begin, end,
    if, else, case, endcase, casez, casex, default, for, while, repeat,
    forever, do, break, continue, return,
    function, endfunction, task, endtask, automatic, static,
    class, endclass, new, extends, virtual, pure, extern, this, super,
    typedef, struct, packed, enum, union, interface, endinterface, modport,
    fork, join, join_any, join_none, disable, wait, wait_order,
    rand, randc, constraint, randomize, solve, before, inside, dist,
    covergroup, endgroup, coverpoint, cross, bins, binsof, with,
    import, export, ref, const, local, protected, string, int, byte,
    shortint, longint, real, shortreal, time, realtime,
    void, assert, assume, cover, expect, property, sequence, endproperty,
    endsequence, clocking, endclocking, program, endprogram, package,
    endpackage, timeunit, timeprecision, unique, priority, final,
    semaphore, mailbox, event, randcase, randsequence,
  },
  sensitive=true,
  morecomment=[l]{//},
  morecomment=[s]{/*}{*/},
  morestring=[b]",
}

\lstset{
  language=SystemVerilog,
  basicstyle=\ttfamily\footnotesize,
  keywordstyle=\color{keywordcolor}\bfseries,
  commentstyle=\color{commentgreen}\itshape,
  stringstyle=\color{stringcolor},
  numberstyle=\tiny\color{numbercolor},
  backgroundcolor=\color{backgroundcolor},
  frame=single,
  rulecolor=\color{black!30},
  numbers=left,
  numbersep=5pt,
  tabsize=2,
  breaklines=true,
  breakatwhitespace=false,
  showstringspaces=false,
  captionpos=b,
  xleftmargin=10pt,
  xrightmargin=5pt,
  aboveskip=8pt,
  belowskip=8pt,
  keepspaces=true,
  columns=flexible
}

% Colored boxes
\newenvironment{coloredbox}[2]{%
  \par\medskip\noindent
  \begin{minipage}{\linewidth}
  \setlength{\fboxsep}{8pt}%
  \fcolorbox{#1!75!black}{#1}{%
  \begin{minipage}{0.96\linewidth}%
  \textbf{#2}\\[3pt]%
}{%
  \end{minipage}}%
  \end{minipage}%
  \medskip\par
}

\newenvironment{exercisebox}{\begin{coloredbox}{exercisecolor}{Exercise}}{\end{coloredbox}}
\newenvironment{solutionbox}{\begin{coloredbox}{solutioncolor}{Solution}}{\end{coloredbox}}
\newenvironment{warningbox}{\begin{coloredbox}{warningcolor}{Warning}}{\end{coloredbox}}
\newenvironment{tipbox}{\begin{coloredbox}{tipcolor}{Tip}}{\end{coloredbox}}
\newenvironment{notebox}{\begin{coloredbox}{notecolor}{Note}}{\end{coloredbox}}
\newenvironment{cheatbox}{\begin{coloredbox}{cheatcolor}{Cheat Sheet}}{\end{coloredbox}}

% Hyperref setup
\hypersetup{
  colorlinks=true,
  linkcolor=blue,
  filecolor=magenta,
  urlcolor=cyan,
  pdftitle={Complete SystemVerilog Guide: Beginner to Advanced},
  pdfauthor={},
  pdfsubject={SystemVerilog Programming and Verification},
  pdfkeywords={SystemVerilog, Verilog, HDL, Verification, UVM, OOP},
  bookmarksnumbered=true,
}

% Header and footer
\pagestyle{fancy}
\fancyhf{}
\fancyhead[L]{Complete SystemVerilog Guide}
\fancyhead[R]{\thepage}
\fancyfoot[C]{Beginner to Advanced with 100+ Exercises}
\renewcommand{\headrulewidth}{0.4pt}
\renewcommand{\footrulewidth}{0.4pt}

% Title formatting
\titleformat{\section}
  {\normalfont\LARGE\bfseries\color{blue!70!black}}
  {\thesection}{1em}{}
\titleformat{\subsection}
  {\normalfont\Large\bfseries\color{blue!50!black}}
  {\thesubsection}{1em}{}
\titleformat{\subsubsection}
  {\normalfont\large\bfseries\color{blue!30!black}}
  {\thesubsubsection}{1em}{}

% Title
\title{
  \vspace{-1.5cm}
  \Huge\textbf{Complete SystemVerilog Guide} \\
  \LARGE From Beginner to Advanced \\
  \vspace{0.3cm}
  \Large With 100+ Exercises, Waveforms, and Cheat Sheets \\
  \vspace{0.3cm}
  \large Master SystemVerilog for Design and Verification
}
\author{}
\date{\today}

\begin{document}

\maketitle

\begin{abstract}
This is the most comprehensive SystemVerilog learning resource available, designed to take you from absolute beginner to expert level. Unlike other materials, this guide includes:

\begin{multicols}{2}
\begin{itemize}[leftmargin=*]
  \item \textbf{Complete Coverage}: All SystemVerilog concepts from basic to advanced
  \item \textbf{100+ Exercises}: Hands-on practice with detailed solutions
  \item \textbf{Waveform Diagrams}: Visual understanding of timing and behavior
  \item \textbf{Cheat Sheets}: Quick reference for every major topic
  \item \textbf{Progressive Learning}: Carefully structured from basics to expert level
  \item \textbf{Real-World Examples}: Practical design and verification projects
  \item \textbf{Best Practices}: Industry-standard coding guidelines
  \item \textbf{Verification Focus}: OOP, randomization, coverage, assertions
  \item \textbf{Complete Solutions}: Detailed explanations for all exercises
  \item \textbf{Self-Assessment}: Quizzes after each major section
\end{itemize}
\end{multicols}

\textbf{Total Learning Time}: 60-80 hours for complete mastery

\textbf{Who This Is For}: Students, verification engineers, design engineers, anyone wanting to master SystemVerilog
\end{abstract}

\tableofcontents
\newpage

% ============================================================================
\section{Introduction and Study Guide}

\subsection{How to Use This Guide}

This comprehensive guide is organized into three main levels:

\begin{enumerate}
  \item \textbf{BEGINNER (Sections 2-10)}: Basic syntax, data types, modules, testbenches
  \item \textbf{INTERMEDIATE (Sections 11-20)}: Advanced features, interfaces, packages, OOP basics
  \item \textbf{ADVANCED (Sections 21-30)}: Verification, randomization, assertions, coverage, UVM
\end{enumerate}

\subsection{Learning Path}

\begin{table}[H]
\centering
\small
\begin{tabular}{|l|l|l|l|}
\hline
\textbf{Level} & \textbf{Sections} & \textbf{Time} & \textbf{Exercises} \\
\hline
Beginner & 2-10 & 20-25 hours & 1-30 \\
Intermediate & 11-20 & 20-25 hours & 31-70 \\
Advanced & 21-30 & 20-30 hours & 71-100 \\
\hline
\textbf{Total} & \textbf{All} & \textbf{60-80 hours} & \textbf{100+} \\
\hline
\end{tabular}
\caption{Recommended Learning Path}
\end{table}

\subsection{Prerequisites}

\textbf{Required:}
\begin{itemize}
  \item Basic understanding of digital logic (AND, OR, flip-flops, etc.)
  \item Familiarity with binary, hexadecimal number systems
  \item Basic programming concepts (variables, loops, conditionals)
\end{itemize}

\textbf{Helpful but not required:}
\begin{itemize}
  \item Verilog experience
  \item C/C++ or any object-oriented programming language
  \item Basic understanding of computer architecture
\end{itemize}

\subsection{Tools You'll Need}

\textbf{Simulators (choose one):}
\begin{itemize}
  \item \textbf{Free}: Icarus Verilog + Verilator, EDA Playground (online)
  \item \textbf{Commercial}: ModelSim, VCS, Xcelium, Questa
\end{itemize}

\textbf{Setup Instructions}:
\begin{lstlisting}[language=bash, caption={Installing Icarus Verilog (Linux/Mac)}]
# Ubuntu/Debian
sudo apt-get install iverilog gtkwave

# macOS
brew install icarus-verilog gtkwave

# Test installation
iverilog -v
\end{lstlisting}

\subsection{Document Conventions}

Throughout this guide:
\begin{itemize}
  \item \texttt{Monospace font} indicates code
  \item \textcolor{blue}{Blue links} are clickable cross-references
  \item \textbf{Bold} highlights important concepts
  \item 📝 Exercises are in yellow boxes
  \item ✅ Solutions are in green boxes
  \item ⚠️ Warnings are in red boxes
  \item 💡 Tips are in blue boxes
\end{itemize}

\newpage
% ============================================================================
% PART I: BEGINNER LEVEL
% ============================================================================

\part{Beginner Level}

\section{Introduction to SystemVerilog}

\subsection{What is SystemVerilog?}

SystemVerilog is a hardware description and verification language that extends Verilog. It combines:
\begin{itemize}
  \item \textbf{Hardware Description}: Design digital circuits
  \item \textbf{Verification}: Test and verify designs using advanced features
  \item \textbf{Object-Oriented Programming}: Classes, inheritance, polymorphism
  \item \textbf{Assertions}: Formal property checking
\end{itemize}

\subsection{SystemVerilog vs Verilog vs VHDL}

\begin{table}[H]
\centering
\small
\begin{tabular}{|l|p{4cm}|p{4cm}|p{4cm}|}
\hline
\textbf{Feature} & \textbf{Verilog} & \textbf{SystemVerilog} & \textbf{VHDL} \\
\hline
Year & 1984 & 2005 & 1987 \\
\hline
Typing & Weak & Strong (with logic) & Very strong \\
\hline
OOP & No & Yes & Limited \\
\hline
Verification & Limited & Extensive & Limited \\
\hline
Assertions & No & Yes (SVA) & Limited (PSL) \\
\hline
Learning Curve & Easy & Medium & Steep \\
\hline
Industry Use & Design & Design + Verification & Design \\
\hline
\end{tabular}
\caption{Language Comparison}
\end{table}

\subsection{Your First SystemVerilog Program}

Let's start with the classic "Hello World":

\begin{lstlisting}[caption={Hello World in SystemVerilog}]
module hello_world;
  initial begin
    $display("Hello, SystemVerilog World!");
    $display("Time: %0t", $time);
  end
endmodule
\end{lstlisting}

\textbf{To run:}
\begin{lstlisting}[language=bash]
# Save as hello_world.sv
iverilog -g2012 -o hello hello_world.sv
vvp hello
\end{lstlisting}

\textbf{Output:}
\begin{verbatim}
Hello, SystemVerilog World!
Time: 0
\end{verbatim}

\textbf{Breaking it down:}
\begin{itemize}
  \item \texttt{module ... endmodule}: Defines a module (basic building block)
  \item \texttt{initial}: Executes once at time 0
  \item \texttt{\$display}: Prints to console (like printf in C)
  \item \texttt{\%0t}: Format specifier for time
  \item \texttt{\$time}: System function returning current simulation time
\end{itemize}

\begin{exercisebox}
\textbf{Exercise 1: Your First Program}

Modify the hello world program to:
\begin{enumerate}
  \item Display your name
  \item Display the current date
  \item Add a delay of 10 time units using \texttt{\#10}
  \item Display the time again after the delay
\end{enumerate}
\end{exercisebox}

\begin{exercisebox}
\textbf{Exercise 2: Multiple Displays}

Write a module that displays numbers 1 through 10, each on a new line, with 5 time units delay between each number.
\end{exercisebox}

\newpage
% ============================================================================
\section{Data Types - Part 1: Basic Types}

\subsection{2-State vs 4-State Types}

SystemVerilog has two categories of data types:

\begin{table}[H]
\centering
\begin{tabular}{|l|l|p{6cm}|}
\hline
\textbf{Category} & \textbf{Types} & \textbf{Possible Values} \\
\hline
2-state & bit, byte, int, shortint, longint & 0, 1 \\
\hline
4-state & logic, reg, wire & 0, 1, X (unknown), Z (high-impedance) \\
\hline
\end{tabular}
\caption{Data Type Categories}
\end{table}

\subsection{Integer Types}

\begin{lstlisting}[caption={Integer Data Types}]
module integer_types;
  // 2-state types (faster simulation, less memory)
  bit        single_bit;      // 1 bit: 0 or 1
  byte       my_byte;          // 8 bits, signed (-128 to 127)
  shortint   my_short;         // 16 bits, signed
  int        my_int;           // 32 bits, signed
  longint    my_long;          // 64 bits, signed

  // 2-state unsigned
  bit [7:0]  unsigned_byte;    // 8 bits: 0 to 255

  // 4-state types (can represent X and Z)
  logic      my_logic;         // 1 bit: 0, 1, X, Z
  logic [7:0] byte_logic;      // 8 bits of 4-state logic
  reg [15:0] my_reg;           // 16 bits (legacy Verilog)
  wire [3:0] my_wire;          // 4 bits (for connections)

  initial begin
    my_int = 42;
    my_byte = -10;
    unsigned_byte = 255;
    my_logic = 1'bX;           // Unknown value

    $display("int: %0d", my_int);
    $display("byte: %0d", my_byte);
    $display("unsigned: %0d", unsigned_byte);
    $display("logic: %b", my_logic);
  end
endmodule
\end{lstlisting}

\subsection{Real Types}

\begin{lstlisting}[caption={Floating Point Types}]
module real_types;
  real      temperature;       // Double precision (64-bit)
  shortreal voltage;           // Single precision (32-bit)

  initial begin
    temperature = 98.6;
    voltage = 3.3;

    $display("Temperature: %f", temperature);
    $display("Voltage: %f", voltage);
    $display("Voltage (2 decimals): %.2f", voltage);
  end
endmodule
\end{lstlisting}

\subsection{String Type}

\begin{lstlisting}[caption={String Type}]
module string_example;
  string greeting;
  string name = "Alice";

  initial begin
    greeting = "Hello";
    $display("%s, %s!", greeting, name);

    // String methods
    $display("Length: %0d", name.len());
    $display("Uppercase: %s", name.toupper());
    $display("Character at [0]: %c", name[0]);

    // String concatenation
    greeting = {greeting, " World"};
    $display("%s", greeting);
  end
endmodule
\end{lstlisting}

\subsection{Number Literals}

\begin{lstlisting}[caption={Number Formats}]
module number_formats;
  int decimal, binary, octal, hexadecimal;

  initial begin
    // Different number formats
    decimal = 42;              // Decimal
    binary = 8'b0010_1010;     // 8-bit binary (underscores for readability)
    octal = 8'o52;             // 8-bit octal
    hexadecimal = 8'h2A;       // 8-bit hexadecimal

    $display("Decimal: %0d", decimal);
    $display("Binary: %0d (0b%b)", binary, binary);
    $display("Octal: %0d (0o%o)", octal, octal);
    $display("Hex: %0d (0x%h)", hexadecimal, hexadecimal);

    // Negative numbers
    int neg = -42;
    $display("Negative: %0d", neg);

    // X and Z in 4-state
    logic [3:0] unknown = 4'bXX01;
    logic [3:0] highz = 4'bZZ10;
    $display("Unknown: %b", unknown);
    $display("High-Z: %b", highz);
  end
endmodule
\end{lstlisting}

\subsection{Waveform: 2-State vs 4-State}

\textbf{Timing Diagram:}
\begin{verbatim}
Time:     0   1   2   3   4   5   6   7
          |   |   |   |   |   |   |   |
bit:      0   1   1   0   1   0   0   1    (only 0 and 1)

logic:    0   1   X   Z   1   0   X   1    (0, 1, X, Z allowed)
\end{verbatim}

\begin{cheatbox}
\textbf{Data Types Cheat Sheet}

\textbf{When to use each type:}
\begin{itemize}
  \item \textbf{bit}: Counters, state machines, testbench variables
  \item \textbf{logic}: Signals that might be driven by multiple sources, X/Z needed
  \item \textbf{int}: General purpose integers in testbenches
  \item \textbf{byte}: Small integers, character data
  \item \textbf{string}: Messages, file names, text processing
  \item \textbf{real}: Analog values, timing calculations
\end{itemize}

\textbf{Quick Reference:}
\begin{verbatim}
bit [7:0] data;      // 8-bit unsigned, 2-state
logic [7:0] data;    // 8-bit, 4-state (can be X/Z)
int data;            // 32-bit signed, 2-state
byte data;           // 8-bit signed, 2-state (-128 to 127)
string msg;          // Dynamic string
real value;          // 64-bit floating point
\end{verbatim}
\end{cheatbox}

\begin{exercisebox}
\textbf{Exercise 3: Data Type Practice}

Create a module that declares and initializes:
\begin{enumerate}
  \item An 8-bit unsigned number with value 255
  \item A signed byte with value -50
  \item A 16-bit binary number: 0b1010\_0101\_1100\_0011
  \item A 32-bit hex number: 0xDEADBEEF
  \item A real number: 3.14159
  \item A string with your name
\end{enumerate}
Display all values using appropriate format specifiers.
\end{exercisebox}

\begin{exercisebox}
\textbf{Exercise 4: Number Conversions}

Write a module that:
\begin{enumerate}
  \item Takes a decimal number 170
  \item Displays it in binary, octal, hex formats
  \item Shows the result with different bit widths (8, 16, 32 bits)
\end{enumerate}
\end{exercisebox}

\begin{exercisebox}
\textbf{Exercise 5: String Manipulation}

Create a module that:
\begin{enumerate}
  \item Declares two strings: first name and last name
  \item Concatenates them with a space
  \item Displays the full name in uppercase
  \item Extracts and displays the first character of each name
  \item Displays the total length of the full name
\end{enumerate}
\end{exercisebox}

\newpage
% ============================================================================
\section{Operators}

\subsection{Arithmetic Operators}

\begin{lstlisting}[caption={Arithmetic Operations}]
module arithmetic_ops;
  int a = 20, b = 7;
  int sum, diff, prod, quot, rem;
  int inc, dec;

  initial begin
    sum = a + b;               // Addition: 27
    diff = a - b;              // Subtraction: 13
    prod = a * b;              // Multiplication: 140
    quot = a / b;              // Division: 2 (integer division)
    rem = a % b;               // Modulus: 6

    inc = a++;                 // Post-increment
    dec = --b;                 // Pre-decrement

    $display("a=%0d, b=%0d", a, b);
    $display("Sum: %0d", sum);
    $display("Difference: %0d", diff);
    $display("Product: %0d", prod);
    $display("Quotient: %0d", quot);
    $display("Remainder: %0d", rem);

    // Power operation
    int power = 2 ** 8;        // 2^8 = 256
    $display("2^8 = %0d", power);
  end
endmodule
\end{lstlisting}

\subsection{Relational and Equality Operators}

\begin{lstlisting}[caption={Comparison Operators}]
module comparison_ops;
  int a = 10, b = 20;
  logic [3:0] x = 4'b1010, y = 4'b1010, z = 4'bXX10;

  initial begin
    // Relational
    $display("a < b: %0d", a < b);         // 1 (true)
    $display("a > b: %0d", a > b);         // 0 (false)
    $display("a <= b: %0d", a <= b);       // 1
    $display("a >= b: %0d", a >= b);       // 0

    // Equality (returns 0, 1, or X)
    $display("x == y: %0d", x == y);       // 1
    $display("x != y: %0d", x != y);       // 0

    // Case equality (compares X and Z exactly)
    $display("x === y: %0d", x === y);     // 1
    $display("x === z: %0d", x === z);     // 0 (different X positions)
    $display("x !== z: %0d", x !== z);     // 1
  end
endmodule
\end{lstlisting}

\subsection{Logical and Bitwise Operators}

\begin{lstlisting}[caption={Logical and Bitwise Operations}]
module logical_bitwise;
  bit a = 1, b = 0;
  bit [3:0] x = 4'b1100, y = 4'b1010;

  initial begin
    // Logical operators (return 0 or 1)
    $display("Logical AND: %0d", a && b);   // 0
    $display("Logical OR: %0d", a || b);    // 1
    $display("Logical NOT: %0d", !a);       // 0

    // Bitwise operators (operate on each bit)
    $display("Bitwise AND: %b", x & y);     // 1000
    $display("Bitwise OR: %b", x | y);      // 1110
    $display("Bitwise XOR: %b", x ^ y);     // 0110
    $display("Bitwise NOT: %b", ~x);        // 0011
    $display("Bitwise NAND: %b", ~(x & y)); // 0111

    // Reduction operators (reduce to single bit)
    $display("Reduction AND: %b", &x);      // 0 (not all 1s)
    $display("Reduction OR: %b", |x);       // 1 (at least one 1)
    $display("Reduction XOR: %b", ^x);      // 0 (even parity)
  end
endmodule
\end{lstlisting}

\subsection{Shift Operators}

\begin{lstlisting}[caption={Shift Operations}]
module shift_ops;
  bit [7:0] data = 8'b1100_1010;

  initial begin
    // Logical shifts (fill with 0)
    $display("Original: %b", data);
    $display("Left shift 2: %b", data << 2);    // 0010_1000
    $display("Right shift 2: %b", data >> 2);   // 0011_0010

    // Arithmetic shifts (preserve sign)
    logic signed [7:0] signed_data = -8;
    $display("Signed: %b (%0d)", signed_data, signed_data);
    $display("Arith right shift: %b (%0d)",
             signed_data >>> 2, signed_data >>> 2);
  end
endmodule
\end{lstlisting}

\subsection{Conditional Operator}

\begin{lstlisting}[caption={Ternary Operator}]
module conditional_op;
  int a = 10, b = 20;

  initial begin
    // condition ? true_value : false_value
    int max = (a > b) ? a : b;
    $display("Max of %0d and %0d is %0d", a, b, max);

    // Nested conditional
    int value = 15;
    string category = (value < 10) ? "low" :
                      (value < 20) ? "medium" : "high";
    $display("Value %0d is %s", value, category);
  end
endmodule
\end{lstlisting}

\subsection{Concatenation and Replication}

\begin{lstlisting}[caption={Concatenation and Replication}]
module concat_replicate;
  bit [3:0] a = 4'b1010;
  bit [3:0] b = 4'b0011;

  initial begin
    // Concatenation
    bit [7:0] concat = {a, b};
    $display("Concatenation: %b", concat);      // 10100011

    // Replication
    bit [11:0] rep = {3{a}};
    $display("Replicate 3 times: %b", rep);     // 101010101010

    // Mixed
    bit [15:0] mixed = {2'b11, a, 2{b}};
    $display("Mixed: %b", mixed);               // 1110100011001100

    // Useful for creating masks
    bit [31:0] all_ones = {32{1'b1}};
    $display("All ones: %h", all_ones);         // ffffffff
  end
endmodule
\end{lstlisting}

\subsection{Operator Precedence}

\begin{table}[H]
\centering
\small
\begin{tabular}{|l|l|l|}
\hline
\textbf{Precedence} & \textbf{Operators} & \textbf{Associativity} \\
\hline
Highest & \texttt{()  []  ::  .} & Left to Right \\
\hline
& \texttt{!  \textasciitilde{}  +  -  \&  \textasciitilde{}\&  |  \textasciitilde{}|  \textasciicircum{}  \textasciitilde{}\textasciicircum{}} (unary) & Right to Left \\
\hline
& \texttt{**} & Right to Left \\
\hline
& \texttt{*  /  \%} & Left to Right \\
\hline
& \texttt{+  -} (binary) & Left to Right \\
\hline
& \texttt{<<  >>  <<<  >>>} & Left to Right \\
\hline
& \texttt{<  <=  >  >=} & Left to Right \\
\hline
& \texttt{==  !=  ===  !==} & Left to Right \\
\hline
& \texttt{\&} & Left to Right \\
\hline
& \texttt{\textasciicircum{}  \textasciitilde{}\textasciicircum{}} & Left to Right \\
\hline
& \texttt{|} & Left to Right \\
\hline
& \texttt{\&\&} & Left to Right \\
\hline
& \texttt{||} & Left to Right \\
\hline
Lowest & \texttt{?:} & Right to Left \\
\hline
\end{tabular}
\caption{Operator Precedence (Highest to Lowest)}
\end{table}

\begin{cheatbox}
\textbf{Operators Cheat Sheet}

\textbf{Arithmetic:} \texttt{+  -  *  /  \%  **  ++  --}

\textbf{Relational:} \texttt{<  <=  >  >=}

\textbf{Equality:} \texttt{==  !=  ===  !==}

\textbf{Logical:} \texttt{\&\&  ||  !}

\textbf{Bitwise:} \texttt{\&  |  \textasciicircum{}  \textasciitilde{}}

\textbf{Shift:} \texttt{<<  >>  <<<  >>>}

\textbf{Reduction:} \texttt{\&  |  \textasciicircum{}  \textasciitilde{}\&  \textasciitilde{}|  \textasciitilde{}\textasciicircum{}}

\textbf{Concatenation:} \texttt{\{a, b\}}

\textbf{Replication:} \texttt{\{n\{a\}\}}

\textbf{Conditional:} \texttt{condition ? true\_val : false\_val}
\end{cheatbox}

\begin{exercisebox}
\textbf{Exercise 6: Calculator}

Create a module that implements a simple calculator:
\begin{enumerate}
  \item Declare two 8-bit numbers
  \item Perform all arithmetic operations (+, -, *, /, \%, **)
  \item Display results with labels
  \item Handle division by zero case
\end{enumerate}
\end{exercisebox}

\begin{exercisebox}
\textbf{Exercise 7: Bit Manipulation}

Write a module that demonstrates:
\begin{enumerate}
  \item Setting bit 3 of a byte to 1
  \item Clearing bit 5 of a byte
  \item Toggling bit 2 of a byte
  \item Checking if bit 4 is set
  \item Swapping nibbles of a byte (upper 4 bits with lower 4 bits)
\end{enumerate}
\end{exercisebox}

\begin{exercisebox}
\textbf{Exercise 8: Parity Checker}

Create a module that:
\begin{enumerate}
  \item Takes an 8-bit input value
  \item Calculates even parity using reduction XOR
  \item Calculates odd parity
  \item Displays both parity bits
\end{enumerate}
\end{exercisebox}

\newpage
% ============================================================================
\section{Procedural Blocks}

\subsection{Initial Block}

The \texttt{initial} block executes once at simulation time 0:

\begin{lstlisting}[caption={Initial Block}]
module initial_example;
  int counter;

  initial begin
    counter = 0;
    $display("[%0t] Counter initialized to %0d", $time, counter);

    #10 counter = 10;
    $display("[%0t] Counter is now %0d", $time, counter);

    #20 counter = 30;
    $display("[%0t] Counter is now %0d", $time, counter);

    #10 $display("[%0t] Simulation ending", $time);
  end
endmodule
\end{lstlisting}

\textbf{Output:}
\begin{verbatim}
[0] Counter initialized to 0
[10] Counter is now 10
[30] Counter is now 30
[40] Simulation ending
\end{verbatim}

\subsection{Always Block}

The \texttt{always} block executes repeatedly:

\begin{lstlisting}[caption={Always Block}]
module always_example;
  logic clk = 0;
  int counter = 0;

  // Generate clock
  always #5 clk = ~clk;

  // Counter on clock edge
  always @(posedge clk) begin
    counter = counter + 1;
    $display("[%0t] Counter = %0d", $time, counter);
    if (counter >= 5) $finish;
  end
endmodule
\end{lstlisting}

\subsection{SystemVerilog Always Variants}

SystemVerilog provides specialized always blocks:

\begin{lstlisting}[caption={Always Variants}]
module always_variants;
  logic clk, rst_n;
  logic [3:0] count;
  logic [7:0] data, next_data;
  logic enable;

  // always_ff: For sequential (flip-flop) logic
  always_ff @(posedge clk or negedge rst_n) begin
    if (!rst_n)
      count <= 4'b0;
    else
      count <= count + 1;
  end

  // always_comb: For combinational logic
  always_comb begin
    next_data = data + 1;
  end

  // always_latch: For latches (not recommended in most cases)
  always_latch begin
    if (enable)
      data = next_data;
  end

  // Clock generation
  initial clk = 0;
  always #5 clk = ~clk;

  // Testbench
  initial begin
    rst_n = 0;
    data = 0;
    enable = 0;
    #10 rst_n = 1;
    #50 $finish;
  end
endmodule
\end{lstlisting}

\textbf{Key Differences:}
\begin{itemize}
  \item \texttt{always\_ff}: Synthesizes to flip-flops (sequential logic)
  \item \texttt{always\_comb}: Synthesizes to combinational logic (gates)
  \item \texttt{always\_latch}: Synthesizes to latches (avoid in most designs)
\end{itemize}

\subsection{Sensitivity Lists}

\begin{lstlisting}[caption={Sensitivity Lists}]
module sensitivity_lists;
  logic a, b, c, result;
  logic clk;

  // Specific signals
  always @(a or b or c) begin
    result = a & b & c;
  end

  // Comma-separated (equivalent)
  always @(a, b, c) begin
    result = a & b & c;
  end

  // Wildcard - sensitive to all signals on RHS
  always @(*) begin
    result = a & b & c;
  end

  // Edge-sensitive
  always @(posedge clk) begin
    // Execute on rising edge of clk
  end

  // Multiple edges
  logic rst_n;
  always @(posedge clk or negedge rst_n) begin
    // Execute on rising clk OR falling rst_n
  end
endmodule
\end{lstlisting}

\subsection{Blocking vs Non-Blocking Assignments}

\textbf{Blocking (=):} Executes sequentially
\textbf{Non-blocking (<=):} Executes in parallel

\begin{lstlisting}[caption={Blocking vs Non-Blocking}]
module blocking_vs_nonblocking;
  logic clk = 0;
  always #5 clk = ~clk;

  // Example 1: Blocking (WRONG for sequential logic!)
  int a_block, b_block;
  always @(posedge clk) begin
    a_block = 1;
    b_block = a_block;  // b_block gets 1 immediately
  end

  // Example 2: Non-blocking (CORRECT for sequential logic!)
  int a_nonblock, b_nonblock;
  always @(posedge clk) begin
    a_nonblock <= 1;
    b_nonblock <= a_nonblock;  // b_nonblock gets OLD value of a
  end

  initial begin
    a_block = 0; b_block = 0;
    a_nonblock = 0; b_nonblock = 0;

    #10;
    $display("After 1 clock:");
    $display("  Blocking: a=%0d, b=%0d", a_block, b_block);
    $display("  Non-blocking: a=%0d, b=%0d", a_nonblock, b_nonblock);

    #10 $finish;
  end
endmodule
\end{lstlisting}

\textbf{Output:}
\begin{verbatim}
After 1 clock:
  Blocking: a=1, b=1        (Both updated!)
  Non-blocking: a=1, b=0    (b gets OLD value)
\end{verbatim}

\textbf{Waveform Comparison:}
\begin{verbatim}
Time:      0    10   20
           |    |    |
clk:     __|‾‾|__|‾‾|__

Blocking:
a_block:   0    1    1
b_block:   0    1    1    <- WRONG! Got new value of a

Non-blocking:
a_nonblock: 0   1    1
b_nonblock: 0   0    1    <- CORRECT! Got old value of a
\end{verbatim}

\begin{warningbox}
\textbf{Golden Rules:}
\begin{enumerate}
  \item Use \textbf{non-blocking (<=)} in \texttt{always\_ff} and sequential blocks
  \item Use \textbf{blocking (=)} in \texttt{always\_comb} and combinational blocks
  \item \textbf{Never mix} blocking and non-blocking in the same always block
  \item Use blocking in \texttt{initial} and \texttt{task/function}
\end{enumerate}
\end{warningbox}

\begin{cheatbox}
\textbf{Procedural Blocks Cheat Sheet}

\begin{lstlisting}
// Executes once at time 0
initial begin
  // Use blocking assignments (=)
end

// Executes on every edge/change
always @(sensitivity_list) begin
  // Legacy Verilog style
end

// For flip-flops (use <= non-blocking)
always_ff @(posedge clk) begin
  q <= d;
end

// For combinational logic (use = blocking)
always_comb begin
  sum = a + b;
end

// For latches (avoid!)
always_latch begin
  if (enable) q = d;
end
\end{lstlisting}
\end{cheatbox}

\begin{exercisebox}
\textbf{Exercise 9: Shift Register}

Create an 8-bit shift register using non-blocking assignments:
\begin{enumerate}
  \item Use a clock with 10ns period
  \item Shift data left by 1 bit each clock cycle
  \item Input a 1 at the LSB each cycle
  \item Display the register value each cycle for 10 cycles
\end{enumerate}
\end{exercisebox}

\begin{exercisebox}
\textbf{Exercise 10: Blocking vs Non-blocking Test}

Write a module that demonstrates the difference:
\begin{enumerate}
  \item Create a 4-stage shift register using blocking
  \item Create another 4-stage shift register using non-blocking
  \item Compare outputs after inputting pattern: 1,0,1,1
  \item Show how the blocking version fails
\end{enumerate}
\end{exercisebox}

% Due to length constraints, I'll create a marker here and continue
% The document continues with more sections...

\newpage
\section{Note on Document Length}

This is the beginning of a comprehensive SystemVerilog guide. The complete document would continue with:

\textbf{Remaining Beginner Topics (Sections 5-10):}
\begin{itemize}
  \item Combinational Logic Design
  \item Sequential Logic Design
  \item Modules and Hierarchy
  \item Testbenches and Simulation
  \item Control Flow (if, case, loops)
  \item Basic Arrays
\end{itemize}

\textbf{Intermediate Topics (Sections 11-20):}
\begin{itemize}
  \item Advanced Data Types (struct, enum, union, typedef)
  \item Arrays (dynamic, associative, queues)
  \item Interfaces and Modports
  \item Packages and Import
  \item Generate Blocks
  \item Parameters
  \item Functions and Tasks (detailed)
  \item Clocking Blocks
  \item Classes and OOP Basics
  \item Inheritance and Polymorphism
\end{itemize}

\textbf{Advanced Topics (Sections 21-30):}
\begin{itemize}
  \item Randomization and Constraints
  \item Functional Coverage and Covergroups
  \item Assertions (SVA)
  \item Inter-process Communication
  \item Virtual Interfaces
  \item Factory Pattern
  \item UVM Basics
  \item Complete Verification Projects
  \item Performance and Optimization
  \item Industry Best Practices
\end{itemize}

The complete document would be approximately \textbf{8000-10000 lines} with all 100+ exercises, solutions, waveforms, and cheat sheets.

Would you like me to:
\begin{enumerate}
  \item Continue building the complete document section by section?
  \item Focus on specific topics you need most?
  \item Create this as multiple smaller documents (one per topic)?
\end{enumerate}

\vfill
\begin{center}
\rule{0.5\textwidth}{0.4pt}\\
\Large\textbf{Complete SystemVerilog Guide}\\
\normalsize
Version 1.0 - Beginner Section\\
\vspace{0.3cm}
\textit{Document in Progress}\\
\textit{Last Updated: \today}\\
\end{center}

\end{document}
