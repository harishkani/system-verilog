\documentclass[11pt,a4paper]{article}

% Basic packages only - compatible with most online LaTeX compilers
\usepackage[utf8]{inputenc}
\usepackage[T1]{fontenc}
\usepackage[margin=1in]{geometry}
\usepackage{listings}
\usepackage{xcolor}
\usepackage{hyperref}
\usepackage{fancyhdr}
\usepackage{titlesec}
\usepackage{enumitem}
\usepackage{float}
\usepackage{booktabs}
\usepackage{array}

% Colors for syntax highlighting
\definecolor{commentgreen}{RGB}{34,139,34}
\definecolor{stringcolor}{RGB}{208,76,239}
\definecolor{keywordcolor}{RGB}{0,0,255}
\definecolor{backgroundcolor}{RGB}{248,248,248}
\definecolor{numbercolor}{RGB}{128,128,128}

% Additional colors for boxes
\definecolor{exercisecolor}{RGB}{255,250,205}
\definecolor{solutioncolor}{RGB}{230,255,230}
\definecolor{warningcolor}{RGB}{255,230,230}
\definecolor{tipcolor}{RGB}{230,240,255}

% Listings configuration for SystemVerilog
\lstdefinelanguage{SystemVerilog}{
  alsoletter={@},
  morekeywords={
    module, endmodule, input, output, inout, wire, reg, logic, bit,
    always, initial, begin, end, if, else, case, endcase, for, while,
    function, endfunction, task, endtask, return, automatic, static,
    class, endclass, new, extends, virtual, pure, extern, this,
    typedef, struct, enum, union, interface, endinterface,
    fork, join, join_any, join_none, disable,
    rand, randc, constraint, randomize, covergroup, endgroup, coverpoint,
    bins, import, export, ref, const, local, protected, string, int,
    real, void, assert, assume, cover, property, sequence,
    clocking, endclocking, program, endprogram, package, endpackage
  },
  sensitive=true,
  morecomment=[l]{//},
  morecomment=[s]{/*}{*/},
  morestring=[b]",
}

\lstset{
  language=SystemVerilog,
  basicstyle=\ttfamily\small,
  keywordstyle=\color{keywordcolor}\bfseries,
  commentstyle=\color{commentgreen}\itshape,
  stringstyle=\color{stringcolor},
  numberstyle=\tiny\color{numbercolor},
  backgroundcolor=\color{backgroundcolor},
  frame=single,
  rulecolor=\color{black!30},
  numbers=left,
  numbersep=8pt,
  tabsize=4,
  breaklines=true,
  breakatwhitespace=false,
  showstringspaces=false,
  captionpos=b,
  xleftmargin=15pt,
  xrightmargin=5pt,
  aboveskip=10pt,
  belowskip=10pt,
  keepspaces=true,
  columns=flexible
}

% Simple box environments (without tcolorbox)
\newenvironment{colorbox}[1]{
  \par\medskip
  \noindent
  \begin{minipage}{\linewidth}
  \setlength{\fboxsep}{10pt}
  \colorlet{currentcolor}{#1}
  \fcolorbox{currentcolor!75!black}{currentcolor}{
  \begin{minipage}{0.95\linewidth}
}{
  \end{minipage}
  }
  \end{minipage}
  \medskip
}

\newenvironment{exercisebox}{
  \begin{colorbox}{exercisecolor}
  \textbf{Exercise}\\
}{
  \end{colorbox}
}

\newenvironment{solutionbox}{
  \begin{colorbox}{solutioncolor}
  \textbf{Solution}\\
}{
  \end{colorbox}
}

\newenvironment{warningbox}{
  \begin{colorbox}{warningcolor}
  \textbf{Warning}\\
}{
  \end{colorbox}
}

\newenvironment{tipbox}{
  \begin{colorbox}{tipcolor}
  \textbf{Tip}\\
}{
  \end{colorbox}
}

% Hyperref setup
\hypersetup{
  colorlinks=true,
  linkcolor=blue,
  filecolor=magenta,
  urlcolor=cyan,
  pdftitle={SystemVerilog Functions and Tasks: Complete Guide},
  pdfauthor={},
  pdfsubject={SystemVerilog Programming},
  pdfkeywords={SystemVerilog, Functions, Tasks, HDL, Verification},
  bookmarksnumbered=true,
}

% Header and footer
\pagestyle{fancy}
\fancyhf{}
\fancyhead[L]{SystemVerilog Functions \& Tasks}
\fancyhead[R]{\thepage}
\fancyfoot[C]{Complete Guide}
\renewcommand{\headrulewidth}{0.4pt}
\renewcommand{\footrulewidth}{0.4pt}

% Title formatting
\titleformat{\section}
  {\normalfont\Large\bfseries\color{blue!70!black}}
  {\thesection}{1em}{}
\titleformat{\subsection}
  {\normalfont\large\bfseries\color{blue!50!black}}
  {\thesubsection}{1em}{}
\titleformat{\subsubsection}
  {\normalfont\normalsize\bfseries\color{blue!30!black}}
  {\thesubsection}{1em}{}

% Title
\title{
  \vspace{-2cm}
  \Huge\textbf{SystemVerilog Functions and Tasks} \\
  \LARGE Complete Learning Guide \\
  \Large From Beginner to Expert
}
\author{}
\date{\today}

\begin{document}

\maketitle

\begin{abstract}
This comprehensive guide is designed to take you from a complete beginner to an expert in SystemVerilog functions and tasks. This guide includes:
\begin{itemize}
  \item Progressive learning from basic to advanced concepts
  \item Hands-on exercises with detailed solutions
  \item Real-world examples including complete testbenches and protocol drivers
  \item Self-assessment quizzes
  \item Troubleshooting guide with common errors
  \item Quick reference cheat sheet
\end{itemize}
This is a complete learning resource suitable for self-study, classroom instruction, or as a professional reference.
\end{abstract}

\tableofcontents
\newpage

% ============================================================================
\section{Introduction}

\subsection{What Are Functions and Tasks?}

Functions and tasks are fundamental building blocks in SystemVerilog that enable:
\begin{itemize}
  \item \textbf{Code Reusability}: Write once, use many times
  \item \textbf{Modularity}: Break complex operations into manageable pieces
  \item \textbf{Abstraction}: Hide implementation details
  \item \textbf{Maintainability}: Easier to debug and update
\end{itemize}

\subsection{Key Differences: Functions vs Tasks}

\begin{table}[H]
\centering
\begin{tabular}{|p{3cm}|p{5cm}|p{5cm}|}
\hline
\textbf{Feature} & \textbf{Functions} & \textbf{Tasks} \\
\hline
Return Value & Must return a value & Cannot return a value (use output/inout) \\
\hline
Timing Control & No delays allowed & Can contain delays (\#, @, wait) \\
\hline
Execution Time & Zero simulation time & Can consume simulation time \\
\hline
Task/Function Calls & Can call functions only & Can call both tasks and functions \\
\hline
Output Arguments & Single return value (SV allows output/ref) & Multiple outputs via output/inout \\
\hline
Usage in Expressions & Can be used in expressions & Cannot be used in expressions \\
\hline
\end{tabular}
\caption{Functions vs Tasks Comparison}
\end{table}

\begin{tipbox}
\textbf{Quick Decision Guide:}
\begin{itemize}
  \item Need to return a value and use in expression? Use Function
  \item Need timing control or delays? Use Task
  \item Pure computation without side effects? Use Function
  \item Multiple outputs needed? Use Task or Function with output/ref
\end{itemize}
\end{tipbox}

% ============================================================================
\section{Beginner Level - Functions}

\subsection{Your First Function}

Let's start with the simplest possible function:

\begin{lstlisting}[caption={The Simplest Function}]
function int add(int a, int b);
    return a + b;
endfunction
\end{lstlisting}

\textbf{Breaking it down:}
\begin{itemize}
  \item \texttt{function} - Keyword to declare a function
  \item \texttt{int} - Return type (32-bit signed integer)
  \item \texttt{add} - Function name
  \item \texttt{(int a, int b)} - Parameters: two integers
  \item \texttt{return a + b} - Returns the sum
  \item \texttt{endfunction} - Marks the end
\end{itemize}

\subsection{Using Functions}

\begin{lstlisting}[caption={Using the add Function}]
module test_add;
    int result;

    initial begin
        result = add(5, 3);
        $display("5 + 3 = %0d", result);

        // Functions can be used directly in expressions
        $display("10 + 20 = %0d", add(10, 20));

        // Functions can be nested
        $display("Nested: %0d", add(add(1, 2), add(3, 4)));
    end
endmodule
\end{lstlisting}

\begin{exercisebox}
\textbf{Exercise 1: Create a subtract function}

Write a function called \texttt{subtract} that takes two integers and returns their difference. Test it with values: 10 - 3, 100 - 50, 7 - 12.
\end{exercisebox}

\subsection{Function Return Types}

Functions can return various data types:

\begin{lstlisting}[caption={Different Return Types}]
// Bit function - returns single bit
function bit is_even(int n);
    return (n % 2 == 0);
endfunction

// Real function - returns floating point
function real celsius_to_fahrenheit(real celsius);
    return (celsius * 9.0/5.0) + 32.0;
endfunction

// String function - returns string
function string get_grade(int score);
    if (score >= 90) return "A";
    if (score >= 80) return "B";
    if (score >= 70) return "C";
    if (score >= 60) return "D";
    return "F";
endfunction
\end{lstlisting}

\subsection{Automatic Functions (For Recursion)}

The \texttt{automatic} keyword creates new storage for each function call:

\begin{lstlisting}[caption={Recursive Factorial Function}]
function automatic int factorial(int n);
    if (n <= 1)
        return 1;
    else
        return n * factorial(n - 1);
endfunction

module test_factorial;
    initial begin
        for (int i = 0; i <= 10; i++)
            $display("%0d! = %0d", i, factorial(i));
    end
endmodule
\end{lstlisting}

\begin{warningbox}
\textbf{Always use \texttt{automatic} for recursive functions!}

Without \texttt{automatic}, variables are shared between recursive calls, leading to incorrect results.
\end{warningbox}

\begin{exercisebox}
\textbf{Exercise 2: Fibonacci sequence}

Write a recursive function \texttt{fibonacci(int n)} that returns the nth Fibonacci number. Formula: fib(0)=0, fib(1)=1, fib(n)=fib(n-1)+fib(n-2)
\end{exercisebox}

% ============================================================================
\section{Beginner Level - Tasks}

\subsection{Your First Task}

Tasks are similar to functions but can contain timing controls:

\begin{lstlisting}[caption={Simple Task}]
task display_message(input string msg);
    $display("[%0t] %s", $time, msg);
endtask

module test_task;
    initial begin
        display_message("Simulation started");
        #10;
        display_message("After 10 time units");
    end
endmodule
\end{lstlisting}

\subsection{Tasks with Output Parameters}

Tasks return values through \texttt{output} parameters:

\begin{lstlisting}[caption={Task with Outputs}]
task add_and_multiply(
    input int a,
    input int b,
    output int sum,
    output int product
);
    sum = a + b;
    product = a * b;
endtask

module test_outputs;
    int s, p;

    initial begin
        add_and_multiply(5, 3, s, p);
        $display("5 + 3 = %0d, 5 * 3 = %0d", s, p);
    end
endmodule
\end{lstlisting}

\subsection{Tasks with Timing}

The key advantage of tasks: they can contain delays!

\begin{lstlisting}[caption={Task with Timing Control}]
task wait_and_print(input int delay_time, input string msg);
    $display("[%0t] Waiting %0d time units...", $time, delay_time);
    #delay_time;
    $display("[%0t] %s", $time, msg);
endtask

module test_timing;
    initial begin
        $display("[%0t] Start", $time);
        wait_and_print(10, "First message");
        wait_and_print(20, "Second message");
        $display("[%0t] Done", $time);
    end
endmodule
\end{lstlisting}

\begin{exercisebox}
\textbf{Exercise 3: Bus write task}

Write a task \texttt{bus\_write(input bit [7:0] addr, input bit [31:0] data)} that:
\begin{enumerate}
  \item Displays "Writing to address..."
  \item Waits 5 time units
  \item Displays "Write complete"
\end{enumerate}
\end{exercisebox}

\subsection{Automatic Tasks (Reentrant)}

Like functions, tasks can be automatic for concurrent execution:

\begin{lstlisting}[caption={Automatic Task for Parallel Execution}]
task automatic delayed_print(input int id, input int delay);
    #delay;
    $display("[%0t] Task %0d completed", $time, id);
endtask

module test_parallel;
    initial begin
        fork
            delayed_print(1, 10);
            delayed_print(2, 5);
            delayed_print(3, 15);
        join
        $display("[%0t] All tasks done", $time);
    end
endmodule
\end{lstlisting}

% ============================================================================
\section{Intermediate Level - Advanced Function Features}

\subsection{Default Arguments}

Functions can have default parameter values:

\begin{lstlisting}[caption={Functions with Default Arguments}]
function int power(int base, int exp = 2);
    int result = 1;
    for (int i = 0; i < exp; i++)
        result *= base;
    return result;
endfunction

module test_defaults;
    initial begin
        $display("3^2 = %0d", power(3));        // Uses default exp=2
        $display("2^8 = %0d", power(2, 8));     // Overrides default
    end
endmodule
\end{lstlisting}

\subsection{Output and Ref Arguments}

Functions can have \texttt{output} and \texttt{ref} parameters:

\begin{lstlisting}[caption={Function with Output Arguments}]
function int divide_with_remainder(
    int dividend,
    int divisor,
    output int remainder
);
    if (divisor == 0) begin
        remainder = 0;
        return 0;
    end
    remainder = dividend % divisor;
    return dividend / divisor;
endfunction
\end{lstlisting}

\subsubsection{Reference Parameters (ref)}

\texttt{ref} enables pass-by-reference, avoiding copies of large data:

\begin{lstlisting}[caption={Using ref for Pass-by-Reference}]
function void swap(ref int a, ref int b);
    int temp = a;
    a = b;
    b = temp;
endfunction

module test_ref;
    int x = 10, y = 20;

    initial begin
        $display("Before swap: x=%0d, y=%0d", x, y);
        swap(x, y);
        $display("After swap: x=%0d, y=%0d", x, y);
    end
endmodule
\end{lstlisting}

\begin{tipbox}
\textbf{When to use ref:}
\begin{itemize}
  \item Large data structures (arrays, structs)
  \item Need to modify the original variable
  \item Performance-critical code (avoids copying)
\end{itemize}
\end{tipbox}

\subsection{Class Methods}

Functions and tasks can be class members:

\begin{lstlisting}[caption={Class Methods Example}]
class BankAccount;
    local real balance;

    function new(real initial_balance = 0);
        balance = initial_balance;
    endfunction

    function void deposit(real amount);
        if (amount > 0) begin
            balance += amount;
            $display("Deposited $%0.2f, Balance: $%0.2f",
                     amount, balance);
        end
    endfunction

    function bit withdraw(real amount);
        if (amount > balance) begin
            $display("Error: Insufficient funds");
            return 0;
        end else begin
            balance -= amount;
            $display("Withdrew $%0.2f, Balance: $%0.2f",
                     amount, balance);
            return 1;
        end
    endfunction

    function real get_balance();
        return balance;
    endfunction
endclass
\end{lstlisting}

% ============================================================================
\section{Advanced Level - Virtual Functions}

\subsection{Understanding Polymorphism}

Virtual functions enable polymorphism in SystemVerilog:

\begin{lstlisting}[caption={Polymorphism Example}]
// Base class
class Packet;
    rand bit [7:0] payload[];

    virtual function void display();
        $display("Generic Packet: %p", payload);
    endfunction
endclass

// Derived class
class EthernetPacket extends Packet;
    bit [47:0] dest_mac;
    bit [47:0] src_mac;

    virtual function void display();
        $display("Ethernet Packet:");
        $display("  Dest MAC: %h", dest_mac);
        $display("  Src MAC: %h", src_mac);
        $display("  Payload: %p", payload);
    endfunction
endclass

module test_polymorphism;
    Packet packets[];
    EthernetPacket ep;

    initial begin
        packets = new[1];
        ep = new();
        ep.dest_mac = 48'hFF_FF_FF_FF_FF_FF;
        ep.payload = '{1, 2, 3, 4};
        packets[0] = ep;

        packets[0].display();  // Calls EthernetPacket::display()
    end
endmodule
\end{lstlisting}

% ============================================================================
\section{Best Practices}

\subsection{Use Automatic for Recursive Code}

Always use \texttt{automatic} for recursive functions:

\begin{lstlisting}[caption={Proper Use of Automatic}]
// GOOD: Automatic for recursion
function automatic int recursive_sum(int n);
    if (n <= 0) return 0;
    return n + recursive_sum(n-1);
endfunction
\end{lstlisting}

\subsection{Prefer Functions Over Tasks When Possible}

Use functions for pure computations:

\begin{lstlisting}[caption={Functions for Pure Computation}]
// GOOD: Function for pure computation
function int max(int a, int b);
    return (a > b) ? a : b;
endfunction
\end{lstlisting}

\subsection{Use Ref for Large Data Structures}

Use \texttt{ref} to avoid copying:

\begin{lstlisting}[caption={Using Ref for Large Structures}]
function void process_array(ref int arr[]);
    foreach(arr[i])
        arr[i] = arr[i] * 2;
endfunction
\end{lstlisting}

% ============================================================================
\section{Common Pitfalls}

\subsection{Timing in Functions}

Functions cannot contain timing controls:

\begin{lstlisting}[caption={Timing Control Error}]
// ERROR: Cannot have delays in functions
// function int bad_function(int value);
//     #10;  // ILLEGAL!
//     return value * 2;
// endfunction

// CORRECT: Use task for timing
task delayed_operation(input int value, output int result);
    #10;
    result = value * 2;
endtask
\end{lstlisting}

\subsection{Forgetting Automatic for Recursion}

\begin{lstlisting}[caption={Recursion Requires Automatic}]
// CORRECT: Automatic recursion
function automatic int fibonacci(int n);
    if (n <= 1) return n;
    return fibonacci(n-1) + fibonacci(n-2);
endfunction
\end{lstlisting}

% ============================================================================
\section{Exercise Solutions}

\begin{solutionbox}
\textbf{Solution to Exercise 1: Subtract function}

\begin{lstlisting}
function int subtract(int a, int b);
    return a - b;
endfunction

module test_subtract;
    initial begin
        $display("10 - 3 = %0d", subtract(10, 3));
        $display("100 - 50 = %0d", subtract(100, 50));
        $display("7 - 12 = %0d", subtract(7, 12));
    end
endmodule
\end{lstlisting}
\end{solutionbox}

\begin{solutionbox}
\textbf{Solution to Exercise 2: Fibonacci sequence}

\begin{lstlisting}
function automatic int fibonacci(int n);
    if (n <= 0) return 0;
    if (n == 1) return 1;
    return fibonacci(n-1) + fibonacci(n-2);
endfunction

module test_fibonacci;
    initial begin
        for (int i = 0; i <= 10; i++)
            $display("fib(%2d) = %5d", i, fibonacci(i));
    end
endmodule
\end{lstlisting}
\end{solutionbox}

% ============================================================================
\section{Quick Reference Guide}

\begin{table}[H]
\centering
\small
\begin{tabular}{|l|l|p{5cm}|}
\hline
\textbf{Feature} & \textbf{Syntax} & \textbf{Use Case} \\
\hline
Basic Function & \texttt{function int f(int a);} & Return single value \\
\hline
Void Function & \texttt{function void f();} & Side effects only \\
\hline
Automatic Function & \texttt{function automatic ...} & Recursion \\
\hline
Default Args & \texttt{function f(int a=0);} & Optional parameters \\
\hline
Output Arg & \texttt{function f(output int x);} & Multiple returns \\
\hline
Ref Arg & \texttt{function f(ref int x);} & Modify original \\
\hline
Virtual Function & \texttt{virtual function f();} & Polymorphism \\
\hline
Basic Task & \texttt{task t(input int a);} & Multiple outputs \\
\hline
Task with Timing & \texttt{task t(); \#10; endtask} & Delays \\
\hline
Automatic Task & \texttt{task automatic t();} & Concurrent execution \\
\hline
\end{tabular}
\caption{Quick Syntax Reference}
\end{table}

% ============================================================================
\section{Summary}

\subsection{Key Takeaways}

\begin{enumerate}
    \item Use \textbf{functions} for pure computations, \textbf{tasks} for timing
    \item Always use \textbf{automatic} for recursion
    \item Use \textbf{ref} for large data structures
    \item \textbf{Virtual functions} enable polymorphism
    \item Functions execute in zero time, tasks can have delays
\end{enumerate}

\subsection{Additional Resources}

\begin{itemize}
    \item IEEE 1800-2017 SystemVerilog Standard
    \item \textit{SystemVerilog for Verification} by Chris Spear
    \item \textit{Writing Testbenches using SystemVerilog} by Janick Bergeron
\end{itemize}

% ============================================================================
\vfill
\begin{center}
\rule{0.5\textwidth}{0.4pt}\\
\Large\textbf{SystemVerilog Functions \& Tasks}\\
\normalsize
Complete Learning Guide\\
\vspace{0.5cm}
\textit{Document Version: 2.0 - Simplified}\\
\textit{Last Updated: \today}\\
\end{center}

\end{document}
